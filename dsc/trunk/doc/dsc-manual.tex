\documentclass{report}
\usepackage{psfig}

\def\dsc{{\sc dsc}}


\begin{document}

\begin{titlepage}
\title{DSC Manual}
\author{Duane Wessels}
\date{\today}
\end{titlepage}

\maketitle

\chapter{Introduction}

{\dsc} is a system for collecting, archiving, and displaying statistics from
a DNS server.  

\section{Components}

{\dsc} consists of the following components:

\subsection{The Collector}

The collector is a binary program, named {\tt dsc\/}, which snoops on 
DNS messages.  It is written in C and uses {\em libpcap\/} for packet capture.

{\tt dsc\/} uses a relatively simple configuration file {\em dsc.conf\/} to define certain
parameters and options.  The configuration file also determines the {\em datasets\/}
that {\tt dsc\/} collects.

{\tt dsc\/} dumps the datasets to XML files every 60 seconds.

\subsection{XML Data Transfer}

You may run the {\tt dsc\/} collector on a remote machine.  That
is, the collector may run on a different machine than where the
data is archived and displayed.  {\dsc} includes some Perl and /bin/sh
scripts that enable transporting XML files over a secure HTTP
connection.

To make this work, Apache+mod\_ssl should run on the archive machine.
Data transfer is authenticated via SSL X509 certificates.  A Perl
CGI script handles all PUT requests on the server.  If the client
certificate is allowed, XML files are stored in the appropriate
directory.

A shell script runs on the collector to upload the XML files.  It
uses {\tt curl\/}\footnote{http://curl.haxx.se} to establish an
HTTPS connection.  XML files are bundled together with {\tt tar\/}
before transfer to eliminate per-connection delays.

You could use {\tt scp\/} or {\tt rsync\/}/{\tt ssh\/} instead of
{\tt curl\/} if you like.

\subsection{XML Extractors}

The XML extractors are Perl scripts that read the XML files from
{\tt dsc\/}.  Presently there is one XML extractor Perl script for
each type of dataset.  Thus, if you add a new dataset definition
to {\em dsc.conf\/}, you also need to write a corresponding XML
extractor.

An XML extractor essentially convers the XML-structured data to a
format that is easier (faster) for the graphing tools to parse.
All XML extractors save the data to ``flat'' text files.  Some
extractors also store the data to relational database tables.

\subsection{Ploticus Graphers}

{\dsc} uses {\em Ploticus\/}\footnote{http://ploticus.sourceforge.net/}
as the graphing engine.  The ``Ploticus Graphers'' read the text
files from the XML extractors.  Graphers are also currently
dataset-specific.   Graphers are written in Perl and use a
now-unavailable Perl Module for the interface to Ploticus.


\section{Architecture}

Figure~\ref{fig-architecture} shows the {\dsc} architecture.  

\begin{figure}
\centerline{\psfig{figure=dsc-arch.eps,width=3.5in}}
\label{fig-architecture}
\caption{The {\dsc} architecture.}
\end{figure}

Note that {\dsc} supports the concept of {\em servers\/} and {\em
nodes\/}.  A server is generally a logical service, which may
actually consist of multiple nodes.

The {\tt dsc\/} collector program runs on or near the remote nodes.
Its XML output is transferred to the archive via HTTPS PUTs.

The archiver runs the HTTPS server, the XML extractors, and the
Ploticus graphers.

\chapter{Prerequisites}

\begin{itemize}
\item
	A C/C++ Compiler
\item
	Perl
\item
	Apache with mod\_ssl
\item
	At least the following Perl modules:
	\begin{itemize}
	\item CGI-Untaint (CGI::Untaint)
	\item CGI.pm
	\item (Chart::Ploticus)
	\item File-Spec (File::Spec)
	\item File-Temp (File::Temp)
	\item Geography-Countries (Geography::Countries)
	\item IP-Country (IP::Country)
	\item Lockfile-Simple (LockFile::Simple)
	\item Net-IRR (Net::IRR)
	\item Net-Whois (Net::Whois)
	\item Scalar-List-Utils (List::Util)
	\item URI
	\item XML-Parser (XML::Parser)
	\item XML-Simple (XML::Simple)
	\end{itemize}
\item
	expat (needed by XML::Parser)
\end{itemize}


\chapter{Installation}

At this point, {\dsc} lacks certain things such as a {\tt ./configure\/}
script and many pathnames are currently hard-coded.  The package
also lacks a top-level makefile that supports {\tt make install\/}.
Therefore, we will install it one subdirectory at a time.

\section{On the Collector}

A collector machine needs only the {\em dsc\/} binary, a configuration file,
and a cron job script.

\subsection{\tt src}

This subdirectory contains the C code for the {\tt dsc\/} binary.
You'll need this program on the collector, but not on the archiver.
To compile:

\begin{verbatim}
% cd src
% make
\end{verbatim}

At the end you should have a {\em dsc\/} binary in the current directory.
You can run {\tt make install\/} if you like.  It copies things to
{\em /usr/local/dsc/bin\/} and {\em /usr/local/dsc/etc\/}.
Alternatively you can manually copy the {\em dsc\/} binary and {\em
dsc.conf.sample\/} file to a location of your choosing.

\subsection{\tt cron}

The collector needs only one file from this directory: {\em
push-data.sh\/}.  This script uses {\tt curl\/} to upload XML files
to the archiver via HTTPS.  You may need to modify it (particularly
the pathnames) for your particular installation.

The script creates a tar file of XML files, then transfers them
with {\tt curl\/}.  If the transfer is successful, the XML files
are removed.

The script should probably run every 60 seconds, the same rate at
which {\tt dsc\/} outputs its XML files.  For example:

\begin{verbatim}
* * * * * /usr/local/dsc/libexec/push-data.sh
\end{verbatim}

\section{The Archiver}

\subsection{\tt xml-extractors}


\subsection{\tt ploticus-graphers}

\subsection{\tt HTML}

\subsection{\tt cron}

\subsection{\tt cgi}

\chapter{Running {\dsc}}

\section{On the Collector}

\subsection{dsc.conf}

Before running {\tt dsc\/} you need to create a configuration file.
Note that configuration directive lines are terminated with a semi-colon.
The configuration file currently understands the following directives:

\begin{description}

\item[local\_address]

	Specifies the DNS server's local IP address.  It is used
	to determine the ``direction'' of an IP packet: sending,
	receiving, or other.  You may specify multiple local addresses
	if necessary.

	Example: {\tt local\_address 172.16.0.1;\/}

\item[run\_dir]

	A directory that should become {\tt dsc\/}'s current directory
	after it starts.  XML files will be written here, as will
	any core dumps.

	Example: {\tt run\_dir "/var/run/dsc";\/}

\item[bpf\_program]

	A Berkeley Packet Filter program string.  Normally you
	should leave this unset.  You may use this to further
	restrict the traffic seen by {\tt dsc\/}.  Note that {\tt
	dsc\/} currently has one indexer that looks at all IP
	packets.  If you specify something like {\em udp port 53\/}
	that indexer will not work.

	Note that this directive must go before the {\em interface\/}
	directive because {\tt dsc\/} makes only one pass through
	the configuration file and the BFP filter is set when the
	interface is initialized.

	Example: {\tt bpf\_program "dst host 192.168.1.1";\/}

\item[interface]

	The interface name to sniff packets from.   You may specify multiple
	interfaces.

	Example: {\tt interface fxp0;\/}

\item[bpf\_vlan\_tag\_byte\_order]

	{\tt dsc\/} knows about VLAN tags.  Some operating systems (FreeBSD-4.x) have a bug
	whereby the VLAN tag id is byte-swapped.  Valid values for this directive
	are {\tt host\/} and {\tt net\/} (the default).    Set this to {\tt host\/}
	if you suspect your operating system has the VLAN tag byte order bug.

	Example: {\tt bpf\_vlan\_tag\_byte\_order host;\/}

\item[match\_vlan]

	A list of VLAN identifiers (integers).  If set, only the packets belonging to these
	VLANs are counted.

	Example: {\tt match\_vlan 101 102;\/}

\item[dataset]

	This directive is the hart of {\dsc}.  Please see the following section for its description.

\end{description}

\section{The dataset directive}

A {\em dataset\/} is a 2-D array of counters.  For example, you
might have a dataset with ``Query Type'' along one dimension and
``Query Name Length'' on the other.  The result is a table that
shows the distribution of query name lengths for each query type.
For example:

\vspace{1ex}
\begin{center}
\begin{tabular}{l|rrrrrr}
Len & A & AAAA & A6 & PTR & NS & SOA \\
\hline
$\cdots$ & & & & & \\
11 & 14 & 8 & 7 & 11 & 2 & 0 \\
12 & 19 & 2 & 3 & 19 & 4 & 1 \\
$\cdots$ & & & & & & \\
255 & 0 & 0 & 0 & 0 & 0 & 0 \\
\hline
\end{tabular}
\end{center}
\vspace{1ex}

\noindent
A dataset is defined by the following parameters:
\begin{itemize}
\item A name
\item A protocol layer (IP or DNS)
\item An indexer for the first dimension
\item An indexer for the second dimension
\item Zero or more options
\end{itemize}

\noindent
{\em dsc.conf\/} syntax:

{\tt dataset\/}
{\em name\/}
{\em protocol\/}
{\em Label1:Indexer1\/}
{\em Label2:Indexer2\/}
{\em options\/};
\vspace{2ex}

\subsubsection{Dataset Name}

The dataset name is used in the filename for {\tt dsc\/}'s XML
files.  Although this is an opaque string, the dataset name is also
used as the name for the XML extractors.  That is, if {\dsc} finds
an XML file named {\em 1092180200.foo.xml\/}, it passes that file
to {\em foo-extractor.pl\/}.

\subsubsection{Protocol}

{\dsc} currently knows about two protocol layers: IP and DNS.
On the {\tt dataset\/} line they are written as {\tt ip\/} and {\tt dns\/}.

\subsubsection{Indexers}

An {\em indexer\/} is simply a function that transforms the attributes
of an IP/DNS message into an array index.  For some attributes the
transformation is straightforward.  For example, the ``Query Type''
indexer simply extracts the query type value from a DNS message and
uses this 16-bit value as the array index.

Other attributes are slightly more complicated.  For example, the
``TLD'' indexer extracts the TLD of the QNAME field of a DNS message
and maps it to an integer.  The indexer maintains a simple internal
table of TLD-to-integer mappings.  The actual integer values are
unimportant because the TLD strings, not the integers, appear in
the resulting XML data.

When you specify an indexer on a {\tt dataset\/} line, you must
provide both the name of the indexer and a label.  The Label appears
as an attribute in the XML output.  For example,
Figure~\ref{fig-sample-xml} shows the XML corresponding to this
{\em dataset\/} line:

\begin{verbatim}
dataset the_dataset dns Foo:foo Bar:bar queries-only;
\end{verbatim}

\begin{figure}
\begin{small}\begin{verbatim}
<array name="the_dataset" dimensions="2" start_time="1091663940" ...
  <dimension number="1" type="Foo"/>
  <dimension number="2" type="Bar"/>
  <data>
    <Foo val="1">
      <Bar val="0" count="4"/>
      ...
      <Bar val="100" count="41"/>
    </Foo>
    <Foo val="2">
      ...
    </Foo>
  </data>
</array>
\end{verbatim}\end{small}
\label{fig-sample-xml}
\caption{Sample XML output}
\end{figure}

In theory you are free to choose any label that you like, however,
the XML extractors look for specific labels.  Please use the labels
given for the indexers in Tables~\ref{tbl-dns-indexers}
and~\ref{tbl-ip-indexers}.  Table~\ref{tbl-dns-indexers} shows the
currently-defined indexers for DNS messages.  Similarly,
Table~\ref{tbl-ip-indexers} shows the indexers for IP packets.

\begin{table}
\label{tbl-dns-indexers}
\begin{center}
\begin{tabular}{|lll|}
\hline
Indexer & Label & Description \\
\hline 
certain\_qnames & CertainQnames & Popular query names seen at roots \\
cip4\_net & ClientSubnet & The client's IPv4 /24 subnet \\
client & - & The client's IPv4 address \\
do\_bit & DO & Whether the DO bit is on \\
edns\_version & EDNSVersion & The EDNS version number \\
idn\_qname & IDNQname & If the QNAME is in IDN format \\
msglen & MsgLen & The DNS message length \\
null & All & A ``no-op'' indexer \\
opcode & Opcode & DNS message opcode \\
qclass & - & Query class \\
qnamelen & QnameLen & Length of the query name \\
qtype & Qtype & DNS query type \\
query\_classification & Class & A classification for bogus queries \\
rcode & Rcode & DNS reply code \\
rd\_bit & RD & Check if Recursion Desired bit set \\
tld & TLD & TLD of the query name \\
\hline
\end{tabular}
\caption{DNS message indexers}
\end{center}
\end{table}

\begin{table}
\label{tbl-ip-indexers}
\begin{center}
\begin{tabular}{|lll|}
\hline
Indexer & Label & Description \\
\hline 
ip\_direction & Direction & one of sent, recv, or other \\
ip\_proto & IPProto & IP protocol (icmp, tcp, udp) \\
\hline
\end{tabular}
\caption{IP packet indexers}
\end{center}
\end{table}

\noindent
Here are some longer descriptions of the DNS indexers:

\begin{description}
\item[certain\_qnames]
	This indexer isolates the two most popular query names seen
	by DNS root servers: {\em localhost\/} and {\em
	[a--m].root-servers.net\/}.
\item[cip4\_net]
	Groups DNS messages together by the /24 subnet of the
	client's IPv4 address.  We use this to make datasets with
	large, diverse client populations more manageable and to
	provide a small amount of privacy and anonymization.
\item[client]
	The IPv4 address of the DNS client.
\item[do\_bit]
	This indexer has only two values: 0 or 1.  It indicates
	whether or not the ``DO'' bit is set in a DNS query.  According to
	RFC 2335: {\em Setting the DO bit to one in a query indicates to the server
	that the resolver is able to accept DNSSEC security RRs.}
\item[edns\_version]
	The EDNS version number, if any, in a DNS query.  EDNS
	Version 0 is documented in RFC 2671.
\item[idn\_qname]
	This indexer has only two values: 0 or 1.  It returns 1
	when the first QNAME in the DNS message question section
	is an internationalized domain name (i.e., containing
	non-ASCII characters).  Such QNAMEs begin with the string
	{\tt xn--\/}.  This convention is documented in RFC 3490.
\item[msglen]
	The overall length (size) of the DNS message.
\item[null]
	A ``no-op'' indexer that always returns the same value.
	This can be used to effectively turn the 2-D table into a
	1-D array.
\item[opcode]
	The DNS message opcode is a four-bit field.  QUERY is the
	most common opcode.  Additional currently defined opcodes
	include: IQUERY, STATUS, NOTIFY, and UPDATE.
\item[qclass]
	The DNS message query class (QCLASS) is a 16-bit value.  IN
	is the most common query class.  Additional currently defined
	query class values include: CHAOS, HS, NONE, and ANY.
\item[qnamelen]
	The length of the first (and usually only) QNAME in a DNS
	message question section.  Note this is the ``expanded''
	length if the message happens to take advantage of DNS
	message ``compression.''
\item[qtype]
	The query type (QTYPE) for the first QNAME in the DNS message
	question section.  Well-known query types include: A, AAAA,
	A6, CNAME, PTR, MX, NS, SOA, and ANY.
\item[query\_classification]
	A stateless classification of ``bogus'' queries:
	\begin{itemize}
	\item non-auth-tld: when the TLD is not one of the IANA-approved TLDs.
	\item root-servers.net: a query for a root server IP address.
	\item localhost: a query for the localhost IP address.
	\item a-for-root: an A query for the DNS root (.).
	\item a-for-a: an A query for an IPv4 address.
	\item rfc1918-ptr: a PTR query for an RFC 1918 address.
	\item funny-class: a query with an unknown/undefined query class.
	\item funny-qtype: a query with an unknown/undefined query type.
	\item src-port-zero: when the UDP message's source port equals zero.
	\item malformed: a malformed DNS message that could not be entirely parsed.
	\end{itemize}
\item[rcode]
	The RCODE value in a DNS reply.  The most common response
	codes are 0 (NO ERROR) and 3 (NXDOMAIN). 
\item[rd\_bit]
	This indexer returns 1 if the RD (recursion desired) bit is
	set in the query.  Usually only stub resolvers set the RD bit.
	Usually authoritative servers do not offer recursion to their
	clients.
\item[tld]
	the TLD of the first QNAME in a DNS message's question section.
\end{description}

\noindent
Here are some longer descriptions of the IP indexers:

\begin{description}
\item[ip\_direction]
	One of three values: sent, recv, or else.  Direction is determined
	based on the setting for {\em local\_address\/} in the configuration file.
\item[ip\_proto]
	The IP protocol type, e.g.: tcp, udp, icmp.
\end{description}
\chapter{Bugs}

\begin{itemize}

\item
	Seems too confusing to have an opaque name for indexers in dsc.conf dataset
	line.  The names are pre-determined anyway since they must match what the
	XML extractors look for.

\end{itemize}

\end{document}
